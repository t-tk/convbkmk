\documentclass[autodetect-engine]{jsarticle}

\newif\ifuptexmode
\makeatletter
\if@jsc@uplatex
 \uptexmodetrue
\else
 \uptexmodefalse
\fi

\usepackage[dvips]{hyperref}
\hypersetup{%
 bookmarksnumbered=true,%
 colorlinks=true,%
 setpagesize=false,%
 pdftitle={いろいろ確かめてみる},%
 pdfsubject={hyperref編},%
 pdfauthor={名無 権兵衛},%
 pdfkeywords={TeX; dvips; dvipdfmx; bookmark; hyperref; しおり; pdf}}

\begin{document}
\parindent0mm

dviware: dvips

\ifuptexmode
engine: upLaTeX
\typeout{### engine: upLaTeX ###}
\else
engine: pLaTeX
\typeout{### engine: pLaTeX ###}
\fi

\section{αβγ}
test test.

\section{абв}
test test.

\section{セクション}
test test.
\subsection{サブセクション(括弧)}
test test.

\ifuptexmode

\section{以下は、upLaTeXのみ}
\subsection{いわゆる『JIS2004』: ♡〠☎☃♨①❷ⅳⅤⓐ㋑}
便利な記号がいっぱい。

\subsection{Extension B (BMP外)の文字: 𠘨𤴔𦥑𧾷𩙿など}
𠘨𤴔𦥑𧾷𩙿

\fi

\end{document}
